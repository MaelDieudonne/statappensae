%%%%%%%%%%%%%%%%%%%%%%%%%%%%%%%%%%%%%%%%%
% ENSAE Assignment Title Page 
% LaTeX Template
% Version 1.0 (27/12/12)
%
% This template has been downloaded from:
% http://www.LaTeXTemplates.com
%
% Original author:
% WikiBooks (http://en.wikibooks.org/wiki/LaTeX/Title_Creation)
% Modified by Julien Perrin to fit ENSAE
%
%%%%%%%%%%%%%%%%%%%%%%%%%%%%%%%%%%%%%%%%%

\documentclass[12pt]{article}
\usepackage[french]{babel}
\usepackage[utf8x]{inputenc}
\usepackage{amsmath}
\usepackage{graphicx}
\usepackage{float}
\usepackage{dsfont}
\usepackage{amsfonts}
\usepackage[T1]{fontenc}
\usepackage[colorinlistoftodos]{todonotes}
\usepackage{listings}
\usepackage{csquotes}
\usepackage{biblatex}

\lstset{
    language=R,
    basicstyle=\scriptsize\ttfamily,
    commentstyle=\ttfamily\color{red},
    numbers=left,
    numberstyle=\ttfamily\color{blue}\footnotesize,
    stepnumber=1,
    numbersep=5pt,
    backgroundcolor=\color{white},
    showspaces=false,
    showstringspaces=false,
    showtabs=false,
    frame=single,
    tabsize=2,
    captionpos=b,
    breaklines=true,
    breakatwhitespace=false,
    title=\lstname,
    escapeinside={},
    keywordstyle={},
    morekeywords={}
    }

\setlength {\marginparwidth }{2cm} 

\title{Stat'app conflit politique}
\addbibresource{references.bib}

\begin{document}



\begin{titlepage}

\newcommand{\HRule}{\rule{\linewidth}{0.5mm}} % Defines a new command for the horizontal lines, change thickness here

\center

% \textsc{\LARGE ENSAE}\\[0.5cm]
\includegraphics[scale=1.2]{ensae_logo_dev.png}\\[1cm]
\textsc{\Large Projet de statistiques appliquées}\\[0.5cm]

\HRule \\[0.4cm]
{ \huge \bfseries Propriété immobilière et comportements électoraux}\\[0.4cm]
\emph{\Large Note de synthèse}\\[0.1cm]
\HRule \\[0.6cm]

\begin{minipage}{0.4\textwidth}
\begin{flushleft} \large
\emph{Auteurs}\\
Alexis \textsc{Barrau}\\
Maël \textsc{Dieudonné}\\
Swann \textsc{Maillefert}\\
\end{flushleft}
\end{minipage}\\[1cm]

\begin{minipage}{0.4\textwidth}
\begin{flushleft} \large
\emph{Encadrants}\\
Pauline \textsc{Mendras}\\
Gaston \textsc{Vermersch}\\
\end{flushleft}
\end{minipage}\\[1cm]

 ---------------------------------------------------------------------------------------\\[0.2cm]

{\large \today}\\[2cm]
\vfill

\end{titlepage}



\section{Introduction}

Le vote constitue un problème classique en sciences sociales, depuis les travaux pionniers d’André Siegfried sur la géographie électorale. Les analyses quantitatives ultérieures, inaugurées aux Etats-Unis par Paul Lazarsfeld et son équipe dans les années 1940, puis développées en France à partir des années 1970, ont isolé plusieurs grands déterminants du vote : l’appartenance socioprofessionnelle et religion d’abord, puis l’âge, le sexe, le degré d’instruction, le lieu de résidence, etc. Schématiquement, les catholiques et les indépendants votent à droite, les salariés (dont particulièrement les ouvriers et les fonctionnaires) votent à gauche, tandis que les jeunes, les chômeurs et les peu diplômés s’abstiennent \cite{douillet_sociologie_2023}. Les effets de ces variables sont toutefois interactifs, évoluent parfois au cours du temps, et sont conditionnés par l’offre électorale, ce qui s’oppose à une lecture trop mécaniste des pratiques de vote.

Le statut de propriétaire immobilier pourrait constituer un déterminant supplémentaire. 
\begin{itemize}
\item Historiquement, l’accession à la propriété apparaît depuis au moins un siècle comme un moyen de stabiliser le vote des classes populaires, en l’arrachant aux forces réactionnaires ou révolutionnaires \cite{michel_cause_2006}.
\item Dès leur origine à la fin du XIXe siècle, les politiques du logement soutiennent l’accession à la propriété (lois Siegfried en 1894, Ribot en 1908, Loucheur en 1928, etc.).
\item Accélération dans les années 1970, aboutissant à une explosion du taux de propriétaires.
\item Dernière évolution notable : depuis 2000, inflation extraordinaire des prix du logement dans les métropoles, qui se répercute progressivement aux alentours.
\item Alors que depuis 1965, ils étaient restés alignés sur le coût de la vie.
\end{itemize}

Le patrimoine est apparu plus récemment comme un déterminant du vote. Analysant les résultats des législatives de 1978, \cite{capdevielle_france_1998} isolent une corrélation entre la détention d’un capital important et le vote à droite. Cet « effet patrimoine » expliquerait par exemple la différence entre agriculteurs possédants, qui votent à droite, et professeurs riches, qui votent à gauche, en dépit de revenus pourtant comparables. Le patrimoine pourrait aussi éclairer l’effet de l’âge : les personnes plus âgées votent plus à droite car elles en possèdent davantage.

\section{Références}

\printbibliography

\end{document}